\documentclass[uplatex,dvipdfmx]{jsarticle}

\usepackage[T1]{fontenc}
\usepackage{textcomp}
\usepackage[dvipdfmx]{graphicx}
\usepackage{amsmath,amssymb}
\usepackage{linguex}
\usepackage{cgloss4e}
\usepackage{bm}
\usepackage{here}
\usepackage{indentfirst}
\usepackage{multirow}
\usepackage{multicol}
\usepackage{comment}
\usepackage{plautopatch}  % Added to ensure proper Japanese font handling
% \usepackage{listings, jvlisting} %ソースコードを書けるようにする
% \usepackage{jlisting} %日本語が文字化けしないようにする
\usepackage{booktabs}
\usepackage{subcaption}

\newcommand{\bvec}[1]{\mbox{\boldmath $#1$}}  % ベクトル用太字
\newcommand{\bs}{$\bigstar$}

% ソースコードを書くための設定
\usepackage{listings,jlisting} %日本語のコメントアウトをする場合jvlisting(もしくはjlisting)が必要
%ここからソースコードの表示に関する設定
\lstset{
  basicstyle={\ttfamily},
  identifierstyle={\small},
  commentstyle={\smallitshape},
  keywordstyle={\small\bfseries},
  ndkeywordstyle={\small},
  stringstyle={\small\ttfamily},
  frame={tb},
  breaklines=true,
  columns=[l]{fullflexible},
  numbers=left,
  xrightmargin=0zw,
  xleftmargin=3zw,
  numberstyle={\scriptsize},
  stepnumber=1,
  numbersep=1zw,
  lineskip=-0.5ex
}
\lstdefinestyle{output}{
  basicstyle=\ttfamily\normalsize, % 出力結果用フォント設定
  frame=tb,                        % 上下に横線
  framesep=5pt,                    % 枠と内容の間隔
  breaklines=true,                 % 長い行を折り返す
  backgroundcolor=\color[gray]{0.95}, % 背景を薄いグレーに
  xleftmargin=1em,                 % 左マージン調整
  xrightmargin=1em,                % 右マージン調整
  showstringspaces=false           % 空白を可視化しない
}
\newcommand{\tab}{表-}
\newcommand{\fig}{図-}
\newcommand{\code}{ソースコード}
\usepackage{color}
\usepackage{tablefootnote}

\renewcommand{\lstlistingname}{ソースコード}

\begin{comment}
Multiline comments:
write anything about this document.
\end{comment}

\setcounter{tocdepth}{1}

\begin{document}

\title{\LaTeX による文書作成の基本}
\author{植田真生史}
\date{\today}

\maketitle

本資料は奥村・黒木著『\LaTeX 2$\epsilon$美文書作成入門』$^\text{\cite{latex}}$の内容を要約,一部抜粋して作成したものである.

\vspace{1cm}
\tableofcontents
\clearpage

\section{\TeX とその仲間}
\subsection{\TeX って何?}

\TeX はスタンフォード大学の数学者・コンピュータ科学者であるDonald E.Knuth教授によって作られた組版ソフトである.
組版(typesetting)とは印刷用語で,活字を組んで版(印刷用の板)を作ることを意味する.
\TeX はコンピュータでテキストと図版をうまく配置して,版にあたるもの(PDFまたはPostScriptファイル)を出力するためのソフトである.

\TeX には次のような特徴がある.
\begin{itemize}
  \item オープンソースソフトウェアであるため,無料で入手し,利用できる.
  \item WindowsでもMacやLinuxなどのUNIX系OSでも,まったく同じ動作をする.
  \item テキスト形式で入力するため,普通のテキストエディタで読み書きでき,再利用・データベース化が容易である.
  \item ペアカーニング\footnote{AVやToのような相補的な形の文字を食い込ませる処理.}や孤立行処理\footnote{段落の最初の行だけ,あるいは最後の行だけが別ページになることを抑制する処理.}といった高度な組版技術が組み込まれている.
  \item 数式を美しく出力できる.
\end{itemize}

\subsection{\LaTeX って何?}

\LaTeX はDEC(現HP)のコンピュータ科学者Leslie Lamport氏によって性能強化された\TeX である.
\TeX 同様にオープンソースソフトウェアである.
最初の\LaTeX は1980年代に作られたが,1993年には\LaTeXe という新しい\LaTeX ができ,現在では\LaTeXe が主流となっている.

\LaTeX の特徴は,文章の論理的な構造と視覚的なレイアウトを分けて考えることができるというところにある.
例えば,「はじめに」という節見出しを
\begin{quote}
  \textbackslash section\{はじめに\}
\end{quote}
のように書いておけば,この「はじめに」を紙面上で別途クラスファイル(.cls)やスタイルファイル(.sty)で設定した節見出しのデザインで出力することができる.

さらに,\LaTeX は章・節・図・表・数式・参考文献などの番号を自動的に紐づけてくれる.
そのため,新たに図などを追加したときに,図番号を振り直さなければならないといった手間が省ける.

\subsection{\TeX と日本語}

日本語の文字数は非常に多く,これらを美しく書き並べることは非常に難易度が高い.
具体的には以下のような処理が必要となる.
\begin{itemize}
  \item 句読点,閉じ括弧,中黒,繰り返し,感嘆符,疑問符が行頭に来ないようにする必要がある.また,促音文字(っ),拗音文字(ゃゅょ),長音文字(ー)もなるべき文頭に来てほしくない.
  \item 開き括弧が行末に来ないようにする必要がある.
  \item 括弧類や句読点が行頭や行末に来たときや,括弧類や句読点が連なったとき(「」「」),空白が空きすぎて見えるので,その空白を詰める必要がある.
  \item 段落の最後の行が1文字と句読点だけとなることはなるべく避けたい.
\end{itemize}

これらの処理を行うため,今日広く使われているp\TeX は上記のような"見栄えの悪い"状態を減点方式で評価し,最適化を行うことで優れた日本語組版を実現している.

\clearpage
\section{\LaTeXe の基本}
\subsection{最低限のルール}
\label{sec:basic_rule}

\LaTeXe の文書を作る際の最低限のルールを以下に示す.
\begin{enumerate}
  \item 文書ファイルの拡張子は.texとする.
  \item 文書の最初に以下のように用紙サイズ\footnote{何も指定しない場合はA4になる.}とテンプレートを指定する.
    \begin{quote}
      \textbackslash documentclass[b5paper]\{jsarticle\}
    \end{quote}
  \item 文章を始める位置に以下の命令を書く.
    \begin{quote}
      \textbackslash begin\{document\}
    \end{quote}
  \item 文章の最後に以下の命令を書く.
    \begin{quote}
      \textbackslash end\{document\}
    \end{quote}
  \item 入力しやすいように適宜Enterキーで改行しても構わない.ただし,Enterキーを2回押して,空白行を作ると,改段落してしまうので,注意が必要である.
  \item \tab\ref{tab:special-characters}に示した特殊文字はそのままでは出力できないので,各文字に適した入力を行う.
    \begin{table}[H]
      \caption{\LaTeX における特殊文字}
      \centering
        \begin{tabular}{lcl}
        \hline
        \textbf{読み}       & \textbf{出力} & \textbf{入力} \\ \hline
          hash               & \#            & \textbackslash\# \\ 
          dollar             & \$            & \textbackslash\$ \\ 
          percent            & \%            & \textbackslash\% \\ 
          ampersand          & \&            & \textbackslash\& \\ 
          tilde              & \~{}          & \textbackslash textasciitilde \\ 
          underscore         & \_            & \textbackslash\_ \\ 
          caret              & \^{}          & \textbackslash textasciicircum \\ 
          backslash          & \textbackslash{} & \textbackslash textbackslash (数式モード内では \textbackslash backslash) \\ 
          left brace         & \{            & \textbackslash\{ \\ 
          right brace        & \}            & \textbackslash\} \\ \hline
        \end{tabular}
      \label{tab:special-characters}
      \end{table}
\end{enumerate}

\subsection{ドキュメントクラス}

\ref{sec:basic_rule}で説明したように,\textbackslash documentclass\{$\cdots$\}は文書の種類を指定するもので,文書ファイルの最初に記述する.

波括弧(\{\})の中には\tab\ref{tab:document-classes}のいずれかを指定する\footnote{出版社や学会によって\tab\ref{tab:document-classes}以外のドキュメントクラスが提供されていることがある.}.
articleの類は論文やレポートなどのいくつかの節(section)からなる文書である.
bookやreportの類は,書籍などのいくつかの章(chapter)からなる文書である.
\begin{table}[H]
  \caption{文書クラスの用途と対応表}
  \centering
    \begin{tabular}{c|cccc}
    \hline
      用途      & 欧文 & 和文(旧・横) & 和文(旧・縦) & 和文(新・横) \\ \hline
      論文・レポート    & article      & (u)jarticle             & (u)tarticle             & jsarticle               \\ 
      長い報告書        & report       & (u)jreport              & (u)treport              & ---                     \\ 
      本                & book         & (u)jbook                & (u)tbook                & jsbook                  \\ \hline
    \end{tabular}
  \label{tab:document-classes}
\end{table}

角括弧([])の中にはオプションとして以下のようなものを指定できる.
\paragraph{本文のフォントサイズ\\}

デフォルト値は10ptである.
例えば,以下のようにフォントサイズを指定する.
\begin{quote}
  \textbackslash documentclass[10.5pt]\{jsarticle\}
\end{quote}

\paragraph{用紙サイズ\\}

デフォルトはA4である.
他にも,例えば次のようなサイズを指定することができる.
\begin{table}[H]
  \begin{quote}
    \begin{tabular}{lll}
       - & a4paper & A4版(210mm $\times$ 297mm) \\
       - & a5paper & A4版(148mm $\times$ 210mm) \\
       - & b4paper & A4版(257mm $\times$ 364mm) \\
       - & b5paper & A4版(182mm $\times$ 257mm) \\
       - & papersize & 出力PDFサイズを用紙サイズと合わせる \\
    \end{tabular}    
  \end{quote}
\end{table}
これらのサイズを以下のように指定する.
\begin{quote}
  \textbackslash documentclass[b5paper]\{jsarticle\}
\end{quote}

\paragraph{段組み\\}

2段組みの文章を作る際は,ドキュメントクラスのオプションで次のように指定する.
\begin{quote}
  \textbackslash documentclass[twocolumn]\{jsarticle\}
\end{quote}

\paragraph{複数のオプションの組み合わせ\\}

複数のオプションを組み合わせて指定する場合は,コンマ(,)を間に入れて次のように指定する.
\begin{quote}
  \textbackslash documentclass[b5paper,10.5pt,twocolumn]\{jsarticle\}
\end{quote}

\subsection{プリアンブル}

文書ファイル中で\textbackslash documentclass\{$\cdots$\}と\textbackslash begin\{document\}の間の部分をプリアンブルと呼び,細かい設定を追加していく.
設定には例えば,次のようなものがある.
\begin{table}[H]
  \begin{quote}
    \begin{tabular}{lll}
       - & \textbackslash pagestyle\{empty\} & ページ番号を振らない \\
       - & \textbackslash usepackage\{newtxtext\} & newtxtextパッケージを取り込んで,欧文や数字をTimes系のフォントにする \\
       - & \textbackslash usepackage\{multirow\} & multirowパッケージを取り込んで,表の行結合を可能にする. \\
       - & \textbackslash usepackage\{multicolumn\} & multirowパッケージを取り込んで,表の列結合を可能にする. \\
       - & \textbackslash newcommand\{\textbackslash tab\}\{表-\} & \textbackslash tab と打ち込むと表-のように出力されるコマンドを新規設定 \\
    \end{tabular}    
  \end{quote}
\end{table}
上記の設定の例はごく一部にすぎず,他にも様々な設定を行うことができる.

\subsection{文章の構造}

\LaTeX では文章の構造を明示するコマンドに以下のようなものを用いる.
\begin{table}[H]
  \begin{quote}
    \begin{tabular}{lll}
       - & \textbackslash part\{ぱーと\} & 部見出し"ぱーと" \\
       - & \textbackslash chapter\{ちゃぷたー\} & 章見出し"ちゃぷたー" \\
       - & \textbackslash section\{せくしょん\} & 節見出し"せくしょん" \\
       - & \textbackslash subsection\{さぶせくしょん\} & 小節見出し"さぶせくしょん" \\
       - & \textbackslash subsubsection\{さぶさぶせくしょん\} & 小々節見出し"さぶさぶせくしょん" \\
       - & \textbackslash paragraph\{ぱらぐらふ\} & 段落見出し"ぱらぐらふ" \\
       - & \textbackslash subparagraph\{さぶぱらぐらふ\} & 小段落見出し"さぶぱらぐらふ" \\
    \end{tabular}    
  \end{quote}
\end{table}

\subsection{タイトルと概要}

タイトルを出力するには次の4つの命令を使う.
\begin{table}[H]
  \begin{quote}
    \begin{tabular}{lll}
       - & \textbackslash title\{たいとる\} & 文章のタイトル"たいとる" \\
       - & \textbackslash auther\{山田太郎\} & 著者"山田太郎" \\
       - & \textbackslash date\{?月?日\} & 執筆日"?月?日"\tablefootnote{\textbackslash today を指定すると自動的にコンパイルした日付になる.} \\
       - & \textbackslash maketitle & 上記3つを出力する命令 \\
    \end{tabular}    
  \end{quote}
\end{table}

\subsection{注釈}

プログラミング言語と同様に文書の中に出力されないコメントを入れることができる.
このようなコメントアウトを行うには文頭に\%を書く.
例えば,次のように注釈を入れることができる.
\begin{lstlisting}[caption=コメントアウトの例,label=code:comment]
  グラフより日本では少子高齢化が進んでいることが分かる.
  % 他にもなにかかけるかもしれない.  
\end{lstlisting}
\code\ref{code:comment}は次のように出力される.

\noindent\textbf{出力結果:}\hrulefill\\
  グラフより日本では少子高齢化が進んでいることが分かる.
  % 他にもなにかかけるかもしれない.
\\\noindent\hrulefill  



\subsection{その他の書式設定}

\LaTeX では他にも様々な書式設定を行うことができる.
以下にその一例を示す.

\begin{table}[H]
  \begin{quote}
    \begin{tabular}{lll}
       - & \textbackslash textgt\{ゴシック体\} & "ゴシック体"の部分を\textgt{ゴシック体}にする \\
       - & \textbackslash textbf\{太字\} & "太字"の部分を\textbf{太字}にする \\
       - & \{ \textbackslash Large少し大きな文字\} & "少し大きな文字"の部分を{\Large 少し大きな文字}にする \\
       - & \textbackslash textcolor\{red\}\{赤文字\} & "赤文字"の部分を\textcolor{red}{赤文字}にする\tablefootnote{プリアンブルで,\textbackslash usepackage\{color\} と書いて,colorパッケージを取り込む必要がある.}\\
    \end{tabular}    
  \end{quote}
\end{table}

その他の書式設定についてはネット等に多くの情報があるので,各自調べてほしい.

\subsection{環境} 

\textbackslash\{hoge\} $\cdots$ \textbackslash\{hoge\}のように対になった命令を環境と呼ぶ.
環境の内側では様々な設定が環境の外側と異なり,逆に環境の内側の書式設定は環境の外側に影響を与えない.
環境を使用した場合の入力の例を\code\ref{code:environment}に示す.
\begin{lstlisting}[caption=環境の例,label=code:environment]
  ここは環境の外
  \begin{center}
    ここは環境の内側.
    ここで\tiny 文字を小さくしても
  \end{center}
  環境の外側ではその影響を受けない.
\end{lstlisting}
\code\ref{code:environment}は以下のように出力される.

\noindent\textbf{出力結果:}\hrulefill\\
  ここは環境の外
   \begin{center}
     ここは環境の内側.
     ここで\tiny 文字を小さくしても
   \end{center}
   環境の外側ではその影響を受けない.
\\\noindent\hrulefill 

環境には他にも図を扱うfigure環境や表を扱うtable環境,右寄せにするflushright環境など様々なものがある.

\subsection{箇条書き}

ここでは箇条書きに関する環境を3つ紹介する.

\subsubsection{itemize環境}

文頭に中黒などの記号をつけた箇条書きである.
例えば,,以下のように入力する.
\begin{lstlisting}[caption=itemize環境,label=code:itemize]
  \LaTeX の特徴を箇条書きでまとめる.
  \begin{itemize}
    \item 数式がきれい
    \item 図表番号を自動で振ってくれる
    \item 文章の論理的な構造と視覚的なレイアウトを分けて考えられる
  \end{itemize}
\end{lstlisting}
\code\ref{code:itemize}は以下のように出力される.

\noindent\textbf{出力結果:}\hrulefill\\
  \LaTeX の特徴を箇条書きでまとめる.
  \begin{itemize}
    \item 数式がきれい
    \item 図表番号を自動で振ってくれる
    \item 文章の論理的な構造と視覚的なレイアウトを分けて考えられる
  \end{itemize} 
\noindent\hrulefill 

\subsubsection{enumerate環境}

文頭に番号をつけた箇条書きである.
例えば,以下のように入力する.
\begin{lstlisting}[caption=enumerate環境,label=code:enumerate]
  \LaTeX の特徴を箇条書きでまとめる.
  \begin{enumerate}
    \item 数式がきれい
    \item 図表番号を自動で振ってくれる
    \item 文章の論理的な構造と視覚的なレイアウトを分けて考えられる
  \end{enumerate}
\end{lstlisting}
\code\ref{code:enumerate}は以下のように出力される.
\clearpage
\noindent\textbf{出力結果:}\hrulefill\\
  \LaTeX の特徴を箇条書きでまとめる.
  \begin{enumerate}
    \item 数式がきれい
    \item 図表番号を自動で振ってくれる
    \item 文章の論理的な構造と視覚的なレイアウトを分けて考えられる
  \end{enumerate} 
\noindent\hrulefill 

\subsubsection{description環境}

左寄せ太字で見出しをつけた箇条書きである.
例えば,以下のように入力する.
\begin{lstlisting}[caption=description環境,label=code:description]
  \LaTeX の特徴を箇条書きでまとめる.
  \begin{description}
    \item[利点1] 数式がきれい
    \item[利点2] 図表番号を自動で振ってくれる
    \item[利点3] 文章の論理的な構造と視覚的なレイアウトを分けて考えられる
  \end{description}
\end{lstlisting}
\code\ref{code:description}は以下のように出力される.

\noindent\textbf{出力結果:}\hrulefill\\
  \LaTeX の特徴を箇条書きでまとめる.
  \begin{description}
    \item[利点1] 数式がきれい
    \item[利点2] 図表番号を自動で振ってくれる
    \item[利点3] 文章の論理的な構造と視覚的なレイアウトを分けて考えられる
  \end{description} 
\noindent\hrulefill 

\clearpage
\section{数式の基本}

\subsection{数式環境の種類}

箇条書きと同様に,数式を記述する環境にも様々な種類がある.
% その一例を\tab\ref{tab:math_env}に示す.


% \begin{table}[H]
%   \caption{数式環境の種類}
%   \centering
%     \begin{tabular}{ll}
%       \hline
%       \textbf{環境}          & \textbf{説明} \\ \hline
%       \$\$                  & インライン数式 \\ 
%       equation             & 1行の別行立て数式 \\ 
%       align                & 位置合わせ可能の別行立て数式 \\ 
%       gather               & 中央揃え可能の別行立て数式 \\ 
%       $\sim*$(align*, alignat* ...) & 上の数式で、番号の付かないもの。基本的に不要(解説参照) \\ 
%       split                & 数式環境の中で、1つの式を複数行に分ける \\ 
%       aligned              & alignと同等のものを数式環境の中で使う \\ 
%       gathered             & gatherと同等のものを数式環境の中で使う \\ \hline
%     \end{tabular}
%   \label{tab:math_env}
% \end{table}

\subsubsection{インライン数式}

変数の定義など,本文の文中で数式表現を用いたいときに使用する.
例えば,以下のように入力する.
\begin{lstlisting}[caption=インライン数式,label=code:inline]
  交通流率$q$と交通密度$k$,速度$v$の3変数の間には常に$q = kv$という関係が常に成り立つ.
\end{lstlisting}
\code\ref{code:inline}は以下のように出力される.

\noindent\textbf{出力結果:}\hrulefill\\
交通流率$q$と交通密度$k$,速度$v$の3変数の間には常に$q = kv$という関係が常に成り立つ.
\\\noindent\hrulefill 

\subsubsection{equation環境}

1行の別行立ての数式を作る環境である.
数式番号が不要な場合は,equation*のようにアスタリスク($\ast$)を付ける.
例えば,以下のように入力する.
\begin{lstlisting}[caption=equation環境,label=code:equation]
  交通密度と速度の関係を一次関数とするGreenshieldsモデルでは速度$v$は式(\ref{equ:greenshields})のように表すことができる.
  \begin{equation}
    v = v_f (1 - \frac{k}{k_j})  \label{equ:greenshields}
  \end{equation}
  ここで,$v_f$は自由速度,$k_j$は飽和密度を表す.
  さらに,式(\ref{equ:greenshields})を$q = kv$の関係を用いて変形すると,
  \begin{equation*}
    q = v_f (1 - \frac{k}{k_j}) k 
  \end{equation*}
  を得る.
\end{lstlisting}
\code\ref{code:equation}は以下のように出力される.

\noindent\textbf{出力結果:}\hrulefill\\
  交通密度と速度の関係を一次関数とするGreenshieldsモデルでは速度$v$は式(\ref{equ:greenshields})のように表すことができる.
  \begin{equation}
    v = v_f (1 - \frac{k}{k_j})  \label{equ:greenshields}
  \end{equation}
  ここで,$v_f$は自由速度,$k_j$は飽和密度を表す.
  さらに,式(\ref{equ:greenshields})を$q = kv$の関係を用いて変形すると,
  \begin{equation*}
    q = v_f (1 - \frac{k}{k_j}) k 
  \end{equation*}
  を得る.
\\\noindent\hrulefill 

\subsubsection{gather環境}

複数の数式を中央揃えで出力する場合に用いる.
特定の数式のみについて数式番号が不要な場合は\textbackslash notagを数式の後に書いておく.
例えば,以下のように入力する.
\begin{lstlisting}[caption=gather環境,label=code:gather]
  2つの式を$kv$関係と$qk$関係をまとめて示すと,次のようになる.
  \begin{gather}
    v = v_f (1 - \frac{k}{k_j}) \notag\\
    q = v_f (1 - \frac{k}{k_j}) k \notag
  \end{gather}
\end{lstlisting}
\code\ref{code:gather}は以下のように出力される.

\noindent\textbf{出力結果:}\hrulefill\\
  2つの式を$kv$関係と$qk$関係をまとめて示すと,次のようになる.
  \begin{gather}
    v = v_f (1 - \frac{k}{k_j}) \notag\\
    q = v_f (1 - \frac{k}{k_j}) k \notag
  \end{gather}
\\\noindent\hrulefill 

\subsubsection{align環境}

複数の数式を指定した位置で揃えて出力する場合に用いる.
位置揃えにはアンド(\&)を用いる.
数式番号が不要な場合は,equation*のようにアスタリスク($\ast$)を付ける.
例えば,以下のように入力する.
\begin{lstlisting}[caption=align環境,label=code:align]
  $qk$関係の式を変形すると
  \begin{align*}
    q &= v_f (1 - \frac{k}{k_j}) k \\
    &= v_f (-\frac{1}{k_j} k^2 + k) \\
    &= - \frac{v_f}{k_j} (k - \frac{k_j}{2})^2 + \frac{k_j v_f}{4}
  \end{align*}
\end{lstlisting}
\code\ref{code:align}は以下のように出力される.
\clearpage
\noindent\textbf{出力結果:}\hrulefill\\
  $qk$関係の式を変形すると
  \begin{align*}
    q &= v_f (1 - \frac{k}{k_j}) k \\
    &= v_f (-\frac{1}{k_j} k^2 + k) \\
    &= - \frac{v_f}{k_j} (k - \frac{k_j}{2})^2 + \frac{k_j v_f}{4}
  \end{align*}
\\\noindent\hrulefill 

\subsubsection{数式環境の中で使う環境}

先ほど示した\code\ref{code:align}では,align*のようにして,数式番号を振らなかった.
実は,これはアスタリスク($\ast$)を付けなかった場合には,各行に式番号が振られてしまうためにやむを得ず行った処理である.
では,式変形などで,1つの式が複数行にわたってしまったときに,その複数行の数式にまとめてひとつの式番号をつけるにはどうしたら良いだろうか?
このような場合には,1つの式を複数行に分けることができるsplit環境が有用である.
split環境を用いた入力例を以下に示す.
\begin{lstlisting}[caption=split環境,label=code:split]
  $qk$関係の式を変形すると,式(\ref{equ:greenshields_qk})のようになる.
  \begin{align}
    \begin{split}
      q &= v_f (1 - \frac{k}{k_j}) k \\
      &= v_f (-\frac{1}{k_j} k^2 + k) \\
      &= - \frac{v_f}{k_j} (k - \frac{k_j}{2})^2 + \frac{k_j v_f}{4}  \label{equ:greenshields_qk}
    \end{split}
  \end{align}
\end{lstlisting}
\code\ref{code:split}は以下のように出力される.

\noindent\textbf{出力結果:}\hrulefill\\
  $qk$関係の式を変形すると,式(\ref{equ:greenshields_qk})のようになる.
  \begin{align}
    \begin{split}
      q &= v_f (1 - \frac{k}{k_j}) k \\
      &= v_f (-\frac{1}{k_j} k^2 + k) \\
      &= - \frac{v_f}{k_j} (k - \frac{k_j}{2})^2 + \frac{k_j v_f}{4}  \label{equ:greenshields_qk}
    \end{split}
  \end{align}
\\\noindent\hrulefill 

\subsection{上付き文字,下付き文字}

上付き文字はハット( $\hat{}$ ),下付き文字はアンダーバー(\_)を用いて記述する.
ただし,上付きや下付きにしたい文字が1文字でない場合には,それらを中括弧(\{\})で括る.
上付き文字や下付き文字の具体的な使用方法は\code\ref{code:split}を参照されたい.
ちなみに,面積などの単位も上付き文字を用いて,mm\$ $\hat{}$ 2\$のように書く.
すると,mm$^2$のように出力される.

\subsection{括弧}

\code\ref{code:equation}から\code\ref{code:split}に示したソースコードの出力結果を見ると,$(1 - \cfrac{k}{k_j})$のように括弧のサイズが数式の上下の高さと揃っていない.
このような問題を解決するコマンドが\textbackslash left と \textbackslash rightである.
これら2つのコマンドは必ずセットで用いる.
このコマンドによる記述を用いることで連立方程式のような複数の数式を括弧で括るような出力を行うこともできる.

\textbackslash left と \textbackslash rightを用いて\code\ref{code:equation}を書き直すと以下のようになる.

\begin{lstlisting}[caption=left/rightコマンド,label=code:left_right]
  交通密度と速度の関係を一次関数とするGreenshieldsモデルでは速度$v$は式(\ref{equ:greenshields2})のように表すことができる.
  \begin{equation}
    v = v_f \left(1 - \frac{k}{k_j} \right)  \label{equ:greenshields2}
  \end{equation}
  ここで,$v_f$は自由速度,$k_j$は飽和密度を表す.
  さらに,式(\ref{equ:greenshields})を$q = kv$の関係を用いて変形すると,
  \begin{equation*}
    q = v_f \left(1 - \frac{k}{k_j} \right) k 
  \end{equation*}
  を得る.
\end{lstlisting}
\code\ref{code:left_right}は以下のように出力される.

\noindent\textbf{出力結果:}\hrulefill\\
  交通密度と速度の関係を一次関数とするGreenshieldsモデルでは速度$v$は式(\ref{equ:greenshields2})のように表すことができる.
  \begin{equation}
    v = v_f \left(1 - \frac{k}{k_j} \right)  \label{equ:greenshields2}
  \end{equation}
  ここで,$v_f$は自由速度,$k_j$は飽和密度を表す.
  さらに,式(\ref{equ:greenshields})を$q = kv$の関係を用いて変形すると,
  \begin{equation*}
    q = v_f \left(1 - \frac{k}{k_j} \right) k 
  \end{equation*}
  を得る.
\\\noindent\hrulefill 

\subsection{その他の数式記号}

数式に用いるコマンドは他にも様々なものがある.
以下に一例を示すが,他にも様々なコマンドが存在するので,各自使用時にネット等で調べてほしい.

\paragraph{和記号・積分記号\\}

\textbackslash sum や \textbackslash int を用いることで,$\sum$や$\int$のように出力できる.

\paragraph{分数\\}

\textbackslash frac\{分子\}\{分母\} や\textbackslash cfrac\{分子\}\{分母\}を用いることで,$\frac{\text{分子}}{\text{分母}}$や$\cfrac{\text{分子}}{\text{分母}}$のように出力できる.

\paragraph{ギリシャ文字\\}

ギリシャ文字は特殊な文字であるため,数式環境の中でコマンドを用いて記述する.
具体的には,$\alpha$と出力したければ,\$\textbackslash alpha\$と記述する.

\paragraph{2項演算子\\}

掛け算記号($\times$)や省略記号($\cdots$)が該当する.
掛け算記号は\$ \textbackslash times \$,省略記号は\$ \textbackslash cdots \$のように書く.

\paragraph{関係演算子\\}

不等号($\leq$)やノットイコール($\neq$)が該当する.
不等号は\$ \textbackslash leq \$,ノットイコールは\$ \textbackslash neq \$のように書く.

\clearpage
\section{グラフィック}

\subsection{図の挿入}

\LaTeX における図の挿入は\code\ref{code:fig_basic}のようにプリアンブルにおける設定と本文における記述が必要になる.

\begin{lstlisting}[caption=図の挿入の基本,label=code:fig_basic]
  % プリアンブルの設定
  \documentclass[dvipdfmx]{jsarticle} % dvipdfmxはドライバ名
  \usepackage{graphicx}

  % 本文の設定
  \begin{document}

  ここに本文を書く \\
  \includegraphics[scale=0.2]{fig1.png} \\ % scale=0.3はオプション
  
  ここに本文を書く
  \end{document}
\end{lstlisting}
\code\ref{code:fig_basic}は以下のように出力される.

\noindent\textbf{出力結果:}\hrulefill\\
  % % プリアンブルの設定
  % \documentclass[dvipdfmx]{jsarticle}
  % \usepackage{graphicx}

  % % 本文の設定
  % \begin{document}
  ここに本文を書く \\
  \includegraphics[scale=0.2]{fig1.png} \\
  
  ここに本文を書く 
  % \end{document}
\\\noindent\hrulefill 

\subsubsection{プリアンブルの設定}

プリアンブルでは,ドライバの指定とgraphicxパッケージを取り込む.

ドライバ名の指定はドキュメントクラスのオプションで行う.
ドライバにはdvipdfmxやdvipsなど複数の種類が存在するが,基本はdvipdfmxを使用する.

graphicxパッケージの取り込みは,他のパッケージと同様で,プリアンブルで\textbackslash usepackage\{graphicx\}と記述する.
パッケージの取り込みにおいてオプションを指定することも可能である.
例えば,\textbackslash usepackage[draft]\{graphicx\}のように記述すると,図の部分が枠とファイル名だけになる.
図を多用した文章では,コンパイルに要する時間が長くなることがあり,上記のようなオプションを指定すると,その時間を短縮することができる.
  
\subsubsection{本文における記述}

本文中では\textbackslash includegraphics[オプション]\{ファイル名\}のように指定すると図を挿入することができる.
ファイル名には挿入したい画像のファイル名(もしくは,パス)を記入する.
なお,オプションは省略することができる.

以下ではオプションに指定できるものの例を紹介する.

\paragraph{widthオプション\\}

指定の幅に合わせるように図を拡大または縮小して表示させたい場合は,widthオプションを用いる.
例えば,width=56mmのように指定する.

\paragraph{heightオプション\\}

指定の高さに合わせるように図を拡大または縮小して表示させたい場合は,heightオプションを用いる.
例えば,height=20mmのように指定する.

\paragraph{widthオプションとheightオプションの同時使用\\}

widthオプションとheightオプションを同時に用いると,図の縦横比を変えることができる.
なお,keepaspectratioも記述すると,縦横比を変えずに指定の高さと幅に収まるように図が配置される.

\paragraph{scaleオプション\\}

画像のサイズを拡大縮小することができる.
widthオプションやheightオプションとは異なり,図の大きさを倍数で指定するため,作成した図に10ptで書いていた文字を文書の上で8ptとして表示させたいといったときに便利である.
例えば,scale=0.8のように記述する.

\paragraph{draftオプション\\}

draftと記述すると,パッケージの取り込みのオプションと同様に,図の枠とファイル名が表示される.

\subsection{figure環境}

論文に限らず,あらゆる文章に掲載される図にはキャプションが付けられているが,\code\ref{code:fig_basic}で出力した図には,キャプションがない.
そこで,figure環境を用いて図を表示させる.
figure環境は,例えば\code\ref{code:fig_env}のように記述する.
図のキャプションは\textbackslash caption\{$\cdots$\}の中に書くことで設定できる.

\begin{lstlisting}[caption=figure環境,label=code:fig_env]
  % プリアンブルの設定
  \documentclass[dvipdfmx]{jsarticle} % dvipdfmxはドライバ名
  \usepackage{graphicx}

  % 本文の設定
  \begin{document}
  
  ここに本文を書く \\
  \begin{figure}[H]
    \centering  % 中央揃え
    \includegraphics[scale=0.2]{fig1.png}  
    \caption{図のきゃぷしょん}  % キャプションは図の下側に表示させる
    \label{fig:fig1}  % 相互参照のためのラベル
  \end{figure}

  図-\ref{fig:fig1}が表示される.
  \end{document}
\end{lstlisting}
\code\ref{code:fig_env}は以下のように出力される.

\noindent\textbf{出力結果:}\hrulefill\\
  % % プリアンブルの設定
  % \documentclass[dvipdfmx]{jsarticle}
  % \usepackage{graphicx}

  % % 本文の設定
  % \begin{document}
  ここに本文を書く \\
  \begin{figure}[H]
    \centering  % 中央揃え
    \includegraphics[scale=0.2]{fig1.png}  
    \caption{図のきゃぷしょん}  % キャプションは図の下側に表示させる
    \label{fig:fig1}  % 相互参照のためのラベル
  \end{figure}

  図-\ref{fig:fig1}が表示される.
  % \end{document}
\\\noindent\hrulefill 

\clearpage
\section{表組み}

\subsection{表組みの基本}

表はtabular環境内で作成する.
\code\ref{code:tabular_env}にその一例を示す.
\code\ref{code:tabular_env}によって記述できる表は3行3列で,1列目が左寄せ,2,3列目が右寄せである\footnote{\textbackslash begin\{tabular\}の波括弧(\{\})の中で左寄せ(l),中央揃え(c),右寄せ(r)を指定する.ここで指定する列数と以降で記述する表の中身の列数が揃っていないとエラーになるので注意が必要である.}.
セルとセルの間は \& を書き,新たな行に移るときは\textbackslash \textbackslash を書く.


\begin{lstlisting}[caption=tabular環境,label=code:tabular_env]
  \begin{tabular}{lll} % 左寄せ
    項目 & \LaTeX & Word \\
    数式 & 美しい & 美しくない \\ 
    相互参照 & めっちゃラク & 面倒
  \end{tabular}
\end{lstlisting}
  
\code\ref{code:tabular_env}は以下のように出力される.
\\ \noindent\textbf{出力結果:}\hrulefill \vspace{2mm}\\
  \begin{tabular}{lll}
    項目 & \LaTeX & Word \\
    数式 & 美しい & 美しくない \\ 
    相互参照 & めっちゃラク & 面倒
  \end{tabular}
  \vspace{2mm}
\\\noindent\hrulefill  
  

\subsection{booktabsによる罫線}
\label{sec:booktabs}

\code\ref{code:tabular_env}によって記述した表には罫線が存在しなかった.
\ref{sec:booktabs}と\ref{sec:hline}では表の罫線の引き方について説明する.

まずは,booktabsパッケージを用いた罫線について説明する.
booktabsパッケージはその他のパッケージと同様にプリアンブルにおいて,取り込みが必要である.
\code\ref{code:tabular_env}の例をbooktabsパッケージを用いて書き直すと,\code\ref{code:booktabs}のようになる.
\code\ref{code:booktabs}について,\textbackslash toprule は上端の罫線を引くコマンド,\textbackslash midruleは中間の罫線を引くコマンド,\textbackslash bottomruleは下端の罫線を引くコマンドである.
これらのコマンドにはオプションを設定することもできるので,各自ネット等の情報を参照されたい.

\begin{lstlisting}[caption=tabular環境,label=code:booktabs]
  \documentclass{jsarticle}
  \usepackage{booktabs}
  \begin{document}

  \begin{tabular}{lll} \toprule  % 上端の罫線
    項目 & \LaTeX & Word \\ \midrule  % 中間の罫線
    数式 & 美しい & 美しくない \\ 
    相互参照 & めっちゃラク & 面倒 \\ \bottomrule  % 下端の罫線
  \end{tabular}

  \end{document}
\end{lstlisting}

\code\ref{code:booktabs}は以下のように出力される.
\noindent\textbf{出力結果:}\hrulefill \vspace{2mm}\\
  % \documentclass{jsarticle}
  % \usepackage{booktabs}
  % \begin{document}
  \vspace{2mm}
  \begin{tabular}{lll} \toprule  % 上端の罫線
    項目 & \LaTeX & Word \\ \midrule  % 中間の罫線
    数式 & 美しい & 美しくない \\ 
    相互参照 & めっちゃラク & 面倒 \\ \bottomrule  % 下端の罫線
  \end{tabular}
  % \end{document}
\\\noindent\hrulefill  


\subsection{\LaTeX 標準の罫線}
\label{sec:hline}

ここではパッケージを使用しない\LaTeX 標準の罫線の引き方について説明する.
パッケージを使用しないため,\ref{sec:booktabs}のようなプリアンブルの記述は不要である.
\LaTeX 標準の罫線を使用した基本の表の作成方法を\code\ref{code:hline}に示す.
縦罫線は\textbackslash begin\{tabular\}の波括弧(\{\})の中で縦線(|)を記述することで出力される.
横罫線は\textbackslash hline を記述することで出力される.
なお,\textbackslash hline \textbackslash hline とすると,二重の横罫線を引くことができる.
また,\textbackslash cline\{欄番号 - 欄番号\}とすることで,一部のみの横罫線を引くことができる.

\begin{lstlisting}[caption=tabular環境,label=code:hline]
  \begin{tabular}{l|ll} \hline
    項目 & \LaTeX & Word \\ \hline \hline  % 二重の横罫線
    数式 & 美しい & 美しくない \\ \cline{1-1}
    相互参照 & めっちゃラク & 面倒 \\ \hline
  \end{tabular}
\end{lstlisting}

\code\ref{code:hline}は以下のように出力される.
\\ \noindent\textbf{出力結果:}\hrulefill \vspace{2mm}\\
  \vspace{2mm}
  \begin{tabular}{l|ll} \hline
    項目 & \LaTeX & Word \\ \hline \hline  % 二重の横罫線
    数式 & 美しい & 美しくない \\ \cline{1-1}
    相互参照 & めっちゃラク & 面倒 \\ \hline
  \end{tabular}
  % \end{document}
\\\noindent\hrulefill  

\subsection{table環境}

論文に限らず,あらゆる文章に掲載される表にはキャプションが付けられているが,\code\ref{code:tabular_env}で出力した表には,キャプションがない.
そこで,table環境を用いて表を出力する.
table環境は,例えば\code\ref{code:table_env}のように記述する.
表のキャプションは\textbackslash caption\{$\cdots$\}の中に書くことで設定できる.

\begin{lstlisting}[caption=基本の表,label=code:table_env]
  果物の値段と個数を表-\ref{tab:basic_table}に示す.  
  \begin{table}[h]
    \centering
    \caption{果物の値段と個数}  % ここで表のキャプションを指定する
    \label{tab:basic_table}  % ここで表のラベルを指定する
    \begin{tabular}{l|ll} \hline
      項目 & \LaTeX & Word \\ \hline \hline  % 二重の横罫線
      数式 & 美しい & 美しくない \\ 
      相互参照 & めっちゃラク & 面倒 \\ \hline
    \end{tabular}
  \end{table}
\end{lstlisting}


\noindent\textbf{出力結果:}\hrulefill\\
\LaTeX とWordの比較を表-\ref{tab:basic_table}に示す.  
\begin{table}[h]
  \centering
  \caption{果物の値段と個数}  % ここで表のキャプションを指定する
  \label{tab:basic_table}  % ここで表のラベルを指定する
  \begin{tabular}{l|ll} \hline
    項目 & \LaTeX & Word \\ \hline \hline  % 二重の横罫線
    数式 & 美しい & 美しくない \\ 
    相互参照 & めっちゃラク & 面倒 \\ \hline
  \end{tabular}
\end{table}
\\\noindent\hrulefill  

\subsection{欄内における改行}

表の1つの欄の中に多くの情報を記述しなければならないこともあろう.
\code\ref{code:table_env}で示したような書き方は欄内における改行を認めておらず,多くの情報を1つの欄内に書くことが困難である.
このような場合は\code\ref{code:multiple_lines}に示したように,tabular環境の中にさらにrabular環境を作成することで,エラーを回避し,欄内で改行を行うことができる.

\begin{lstlisting}[caption=基本の表,label=code:multiple_lines]
  \LaTeX とWordの比較を表-\ref{tab:multiple_lines}に示す.  
  \begin{table}[h]
    \centering
    \caption{果物の値段と個数}  % ここで表のキャプションを指定する
    \label{tab:multiple_lines}  % ここで表のラベルを指定する
    \begin{tabular}{l|ll} \hline
      項目 & \LaTeX & Word \\ \hline \hline  % 二重の横罫線
      数式 & 美しい & 美しくない \\ 
      相互参照 & \hspace{-2mm}\begin{tabular}{l}
                  めっちゃラク\\
                  特に図表を追加するときに番号を\\
                  振り直さなくても良いのが素晴らしい.  
                \end{tabular} & 面倒 \\ \hline
    \end{tabular}
  \end{table}
\end{lstlisting}

\code\ref{code:multiple_lines}の出力結果を以下に示す.
\\\noindent\textbf{出力結果:}\hrulefill\\
\LaTeX とWordの比較を表-\ref{tab:multiple_lines}に示す.  
\begin{table}[h]
  \centering
  \caption{果物の値段と個数}  % ここで表のキャプションを指定する
  \label{tab:multiple_lines}  % ここで表のラベルを指定する
  \begin{tabular}{l|ll} \hline
    項目 & \LaTeX & Word \\ \hline \hline  % 二重の横罫線
    数式 & 美しい & 美しくない \\ 
    相互参照 & \hspace{-2mm}\begin{tabular}{l}
                めっちゃラク\\
                特に図表を追加するときに番号を\\
                振り直さなくても良いのが素晴らしい.  
              \end{tabular} & 面倒 \\ \hline
  \end{tabular}
\end{table}
\\\noindent\hrulefill  

% \code\ref{code:multiple_lines}の出力結果を以下に示す.
% \noindent\textbf{出力結果:}\hrulefill\\
% \LaTeX とWordの比較を表-\ref{tab:multiple_lines}に示す.  
% \begin{table}[h]
%   \centering
%   \caption{果物の値段と個数}  % ここで表のキャプションを指定する
%   \label{tab:multiple_lines}  % ここで表のラベルを指定する
%   \hspace{-2mm}\begin{tabular}{l|ll} \hline
%     項目 & \LaTeX & Word \\ \hline \hline  % 二重の横罫線
%     数式 & 美しい & 美しくない \\ 
%     相互参照 & めっちゃラク\par 特に図表を追加するときに番号を\par 振り直さなくても良いのが素晴らしい. & 面倒 \\ \hline
%   \end{tabular}
% \end{table}
% \\\noindent\hrulefill  


\subsection{列割りと行割りの変更}

場合によってはセルの結合を用いて,行割りや列割りを変更することもあろう.
このような場合には,multirowパッケージとmulticolパッケージを用いる.
これらは他のパッケージと同様にプリアンブルにおいてパッケージの取り込みを行うことが必要である.

multicolパッケージを使用した複数列の結合は,tabular環境内で,\textbackslash multicolumn\{結合する列数\}\{文字列の配置\}\{欄内の表記\}のように記述することで実行できる.
multirowパッケージを使用した複数行の結合は,tabular環境内で,\textbackslash multirow\{結合する行数\}\{$\ast$\}\{欄内の表記\}のように記述することで実行できる.

\code\ref{code:multicol}には,multicolパッケージを用いた列の結合の例を示す.
\code\ref{code:multicol}のソースコードはロジットモデルの推定結果を示す表の一例であり,4行目以降で列の結合を行っている.

\begin{lstlisting}[caption=列割りを変更した表,label=code:multicol]
    \documentclass{jsarticle}
    \usepackage{multicol}
    \usepackage{booktabs}
    \begin{document}
    
    \begin{table}[tbph]
      \centering
      \caption{パラメータ推定結果}
      \begin{tabular}{lcc} \toprule
          説明変数 & 推定値 & $t$値 \\\midrule
          説明変数1のパラメータ$\beta_1$ & $\hat{\beta_1}$ & $t_1^*$ \\
          説明変数2のパラメータ$\beta_2$ & $\hat{\beta_2}$ & $t_2^*$ \\
          説明変数3のパラメータ$\beta_3$ & $\hat{\beta_3}$ & $t_3^*$ \\\midrule
          サンプル数 & \multicolumn{2}{c}{N} \\  % 2列を結合して,中央揃えで,Nと記入する.
          尤度比 & \multicolumn{2}{c}{$\rho$} \\
          的中率[\%] & \multicolumn{2}{c}{Acc} \\\bottomrule
      \end{tabular}
      \label{tab:logit_result}
    \end{table}
    
    \end{document}
\end{lstlisting}

\code\ref{code:multicol}の出力結果を以下に示す.\\
\noindent\textbf{出力結果:}\hrulefill\\
% \documentclass{jsarticle}
% \usepackage{multicol}
% \usepackage{booktabs}
% \begin{document}

\begin{table}[tbph]
  \centering
  \caption{パラメータ推定結果}
  \begin{tabular}{lcc} \toprule
      説明変数 & 推定値 & $t$値 \\\midrule
      説明変数1のパラメータ$\beta_1$ & $\hat{\beta_1}$ & $t_1^*$ \\
      説明変数2のパラメータ$\beta_2$ & $\hat{\beta_2}$ & $t_2^*$ \\
      説明変数3のパラメータ$\beta_3$ & $\hat{\beta_3}$ & $t_3^*$ \\\midrule
      サンプル数 & \multicolumn{2}{c}{N} \\
      尤度比 & \multicolumn{2}{c}{$\rho$} \\
      的中率[\%] & \multicolumn{2}{c}{Acc} \\\bottomrule
  \end{tabular}
  \label{tab:logit_result}
\end{table}

% \end{document}
\noindent\hrulefill  \\




\clearpage
\section{図・表の配置}

\subsection{図の自動配置}
\label{sec:allocate_fig}

\code\ref{code:fig_env}において,figure環境を用いた図の表示方法を説明した.
ただし,\code\ref{code:fig_env}の記述方法では図が意図した位置に表示されないことも多い.
そこで,\textbackslash begin\{figure\}のオプションで位置指定を行う.
位置指定に使用できる文字は以下のとおりである.
\begin{itemize}
  \item t:ページ上端(top)に図を出力する
  \item b:ページ下端(bottom)に図を出力する
  \item p:単独ページ(page)に図を出力する
  \item h:できればその位置(here)に図を出力する
\end{itemize}

これらの位置指定オプションは\textbackslash begin\{figure\}[tbph]のように記述して指定する.
このような記述の場合,tでうまく表示できないときにはb,bでうまく表示できないときはpのように,4つの位置指定をその順にうまく表示できるか試行する.
なお,tbpやtbのように一部の文字だけを用いた位置指定でも問題ない.

また,[b!]のように指定すると,[b]よりもより強い指定となる.
さらに,プリアンブルにおいて,\textbackslash usepackage\{float\}と記述して,floatパッケージを取り込むと,[H]のような記述をすることもできる.
この[H]という記述によって,必ずその位置に図を出力するという指定を行うことができる.

\subsection{表の自動配置}

表においても同様である.
\code\ref{code:table_env}の記述方法では表が意図した位置に表示されないことが多いため,\ref{sec:allocate_fig}と同じ位置指定オプションを\textbackslash begin\{table\}のオプションで設定する.

\subsection{左右に並べる配置}

場合によっては複数の画像や表を横並びで表示させたいこともあろう.
その場合は\code\ref{code:minipage_env}のように,minipage環境を用いる.
minipage環境は,\textbackslash begin\{minipage\}\{$\cdots$\}の$\cdots$の部分に記述した幅に合わせた小さなページを作り,その中に図を配置することによって,左右横並びに図を出力させることを可能にする.
指定する幅の大きさによっては,図を横に3枚や4枚並べることも可能になる.

幅の指定はmm等の環境に依存しない指定方法もあるが,\code\ref{code:minipage_env}では,\textbackslash linewidthを用いた環境依存の指定方法を使用している.
\textbackslash linewidthは現在の環境\footnote{記述位置によって異なる.すなわち,\textbackslash begin\{itemize\}と \textbackslash end \{itemize\}の間に書けば, itemize環境が現在の環境である.}の幅に合わせることが可能である.
\code\ref{code:minipage_env}において,0.4\textbackslash linewidthと記述しているのは,minipageの環境の幅を,現在の環境の幅\footnote{ここではdocument環境の幅,すなわち,文章の幅である.}の0.4倍とするためである.
ちなみに,\textbackslash includegraphicsのオプションで,width=\textbackslash linewidthと記述しているが,これは,画像の幅をminipage環境\footnote{これが現在の環境になっている}の幅に合わせるための記述である.

また,複数の図を並べて配置した場合に,それらの下部にキャプションをつけたい場合はsubcaptionパッケージを取り込みが必要である.
プリアンブルにおいて,\textbackslash usepackage\{subcaption\}を記述したうえで,\textbackslash subcaption\{$\cdots$\}にそのキャプションを記述する.

\begin{lstlisting}[caption=minipage環境を使用した配置,label=code:minipage_env]
  \documentclass{jsarticle}
  \usepackage[dvipdfmx]{graphicx}
  \usepackage{subcaption}

  \begin{document}
  
  \begin{figure}[tbph]
    \centering
    \vspace{4mm}
    \begin{minipage}{0.4\linewidth}
        \centering
        \includegraphics[width=\linewidth]{fig_left.jpeg}
        \subcaption{ひだりのきゃぷしょん}
        \label{fig:fig_left}
    \end{minipage}
    \hfill
    \begin{minipage}{0.4\linewidth}
        \centering
        \includegraphics[width=\linewidth]{fig_right.png}
        \subcaption{みぎのきゃぷしょん}
        \label{fig:fig_right}
    \end{minipage}
    \caption{きゃぷしょん}
  \end{figure}

  \end{document}
\end{lstlisting}

\code\ref{code:minipage_env}は以下のように出力される.

\noindent\textbf{出力結果:}\hrulefill\\
\vspace{-6mm}
\begin{figure}[tbph]
  \centering
  \vspace{4mm}
  \begin{minipage}{0.4\linewidth}
      \centering
      \includegraphics[width=\linewidth]{fig_left.jpeg}
      \subcaption{ひだりのきゃぷしょん}
      \label{fig:fig_left}
  \end{minipage}
  \hfill
  \begin{minipage}{0.4\linewidth}
      \centering
      \includegraphics[width=\linewidth]{fig_right.png}
      \subcaption{みぎのきゃぷしょん}
      \label{fig:fig_right}
  \end{minipage}
  \caption{きゃぷしょん}
\end{figure}
\\\noindent\hrulefill  


\clearpage
\section{相互参照}

「\ref{sec:allocate_fig}を参照されたい」や「\fig\ref{fig:fig_left}を参照されたい」のように,章,節,図,表,式などを本文中で引用したいこともあろう.
このときに有用なのが,相互参照機能である.
相互参照機能は\textbackslash labelというコマンドと,\textbackslash refというコマンドをセットで使用する.

まず,\textbackslash label\{$\cdots$\}は引用したい章,節,図,表,式の直後に記述し,\{$\cdots$\}の中に引用時に使用する名称を記述する.
この名称は,自身が分かりやすく,覚えておきやすい名前にしておくと,効率的な執筆ができる.
例えば,\code\ref{code:multicol}では,ロジットモデルの推定結果の表に対して,tab:logit\_resultのようなラベルを付与している.
また,引用対象も明記しておくと良いだろう.
具体的には,節を引用したい場合は\textbackslash label\{sec:hoge\},表を引用したい場合は\textbackslash label\{tab:fuga\}のようなラベルを付与することが望ましい.

上記のように付与したラベルを用いて,本文中で章,節,図,表,式を引用するには\textbackslash refコマンドを使用する.
具体的には,\textbackslash ref\{$\cdots$\}の\{$\cdots$\}に上記の\textbackslash label\{\}コマンドで付与したラベルを記述する.
すなわち,\code\ref{code:multicol}で示したロジットモデルの推定結果の表の場合は,表- \textbackslash ref\{tab:logit\_result\}と記述すると,表-\ref{tab:logit_result}のように出力される.
ちなみに,著者は,"表-"を記述するのが面倒なので,プリアンブルにおいて,\textbackslash newcommand\{\textbackslash tab\}\{表-\}と記述して,\textbackslash tabコマンドを設定している.
これによって,\textbackslash tab\textbackslash ref\{tab:logit\_result\}と記述すると,\tab\ref{tab:logit_result}と出力される.



% 参考文献
\bibliographystyle{junsrt}  % 参考文献を引用された順で出力する
\bibliography{latex}  % main.bib

\end{document}

