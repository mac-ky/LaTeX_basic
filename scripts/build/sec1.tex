\section{\TeX とその仲間}
\subsection{\TeX って何?}

\TeX はスタンフォード大学の数学者・コンピュータ科学者であるDonald E.Knuth教授によって作られた組版ソフトである.
組版(typesetting)とは印刷用語で,活字を組んで版(印刷用の板)を作ることを意味する.
\TeX はコンピュータでテキストと図版をうまく配置して,版にあたるもの(PDFまたはPostScriptファイル)を出力するためのソフトである.

\TeX には次のような特徴がある.
\begin{itemize}
  \item オープンソースソフトウェアであるため,無料で入手し,利用できる.
  \item WindowsでもMacやLinuxなどのUNIX系OSでも,まったく同じ動作をする.
  \item テキスト形式で入力するため,普通のテキストエディタで読み書きでき,再利用・データベース化が容易である.
  \item ペアカーニング\footnote{AVやToのような相補的な形の文字を食い込ませる処理.}や孤立行処理\footnote{段落の最初の行だけ,あるいは最後の行だけが別ページになることを抑制する処理.}といった高度な組版技術が組み込まれている.
  \item 数式を美しく出力できる.
\end{itemize}

\subsection{\LaTeX って何?}

\LaTeX はDEC(現HP)のコンピュータ科学者Leslie Lamport氏によって性能強化された\TeX である.
\TeX 同様にオープンソースソフトウェアである.
最初の\LaTeX は1980年代に作られたが,1993年には\LaTeXe という新しい\LaTeX ができ,現在では\LaTeXe が主流となっている.

\LaTeX の特徴は,文章の論理的な構造と視覚的なレイアウトを分けて考えることができるというところにある.
例えば,「はじめに」という節見出しを
\begin{quote}
  \textbackslash section\{はじめに\}
\end{quote}
のように書いておけば,この「はじめに」を紙面上で別途クラスファイル(.cls)やスタイルファイル(.sty)で設定した節見出しのデザインで出力することができる.

さらに,\LaTeX は章・節・図・表・数式・参考文献などの番号を自動的に紐づけてくれる.
そのため,新たに図などを追加したときに,図番号を振り直さなければならないといった手間が省ける.

\subsection{\TeX と日本語}

日本語の文字数は非常に多く,これらを美しく書き並べることは非常に難易度が高い.
具体的には以下のような処理が必要となる.
\begin{itemize}
  \item 句読点,閉じ括弧,中黒,繰り返し,感嘆符,疑問符が行頭に来ないようにする必要がある.また,促音文字(っ),拗音文字(ゃゅょ),長音文字(ー)もなるべき文頭に来てほしくない.
  \item 開き括弧が行末に来ないようにする必要がある.
  \item 括弧類や句読点が行頭や行末に来たときや,括弧類や句読点が連なったとき(「」「」),空白が空きすぎて見えるので,その空白を詰める必要がある.
  \item 段落の最後の行が1文字と句読点だけとなることはなるべく避けたい.
\end{itemize}

これらの処理を行うため,今日広く使われているp\TeX は上記のような"見栄えの悪い"状態を減点方式で評価し,最適化を行うことで優れた日本語組版を実現している.
