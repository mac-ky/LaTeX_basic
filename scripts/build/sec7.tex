\section{相互参照}

「\ref{sec:allocate_fig}を参照されたい」や「\fig\ref{fig:fig_left}を参照されたい」のように,章,節,図,表,式などを本文中で引用したいこともあろう.
このときに有用なのが,相互参照機能である.
相互参照機能は\textbackslash labelというコマンドと,\textbackslash refというコマンドをセットで使用する.

まず,\textbackslash label\{$\cdots$\}は引用したい章,節,図,表,式の直後に記述し,\{$\cdots$\}の中に引用時に使用する名称を記述する.
この名称は,自身が分かりやすく,覚えておきやすい名前にしておくと,効率的な執筆ができる.
例えば,\code\ref{code:multicol}では,ロジットモデルの推定結果の表に対して,tab:logit\_resultのようなラベルを付与している.
また,引用対象も明記しておくと良いだろう.
具体的には,節を引用したい場合は\textbackslash label\{sec:hoge\},表を引用したい場合は\textbackslash label\{tab:fuga\}のようなラベルを付与することが望ましい.

上記のように付与したラベルを用いて,本文中で章,節,図,表,式を引用するには\textbackslash refコマンドを使用する.
具体的には,\textbackslash ref\{$\cdots$\}の\{$\cdots$\}に上記の\textbackslash label\{\}コマンドで付与したラベルを記述する.
すなわち,\code\ref{code:multicol}で示したロジットモデルの推定結果の表の場合は,表- \textbackslash ref\{tab:logit\_result\}と記述すると,表-\ref{tab:logit_result}のように出力される.
ちなみに,著者は,"表-"を記述するのが面倒なので,プリアンブルにおいて,\textbackslash newcommand\{\textbackslash tab\}\{表-\}と記述して,\textbackslash tabコマンドを設定している.
これによって,\textbackslash tab\textbackslash ref\{tab:logit\_result\}と記述すると,\tab\ref{tab:logit_result}と出力される.
